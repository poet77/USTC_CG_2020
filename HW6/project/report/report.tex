\documentclass{article}
\textheight 23.5cm \textwidth 15.8cm
%\leftskip -1cm
\topmargin -1.5cm \oddsidemargin 0.3cm \evensidemargin -0.3cm
%\documentclass[final]{siamltex}

\usepackage{verbatim}
\usepackage{fancyhdr}
\usepackage{amssymb,ctex}
\usepackage{mathrsfs}
\usepackage{latexsym,amsmath,amssymb,amsfonts,epsfig,graphicx,cite,psfrag}
\usepackage{eepic,color,colordvi,amscd}
\usepackage{enumerate}
\usepackage{enumitem}
\usepackage{booktabs}
\usepackage{graphicx}
\usepackage{float}
\usepackage{wrapfig}
\usepackage{multirow}
\usepackage{subfigure}
\usepackage{diagbox}
\usepackage{wasysym}

\title{USTC\_CG HW6 MassSpring}
\author{张继耀,PB20000204}

\begin{document}
	\maketitle
	
	\tableofcontents
	
	\section {问题介绍}
	\subsection{主要目的}
	
		\begin{itemize}
		\item 在给定的网格框架上完成作业,实现
			\begin{itemize}[label=$\circ$, itemjoin=\hspace{0.5em}]
			\item 质点弹簧仿真模型的欧拉隐式方法
			\item 质点弹簧仿真模型的加速模拟方法
		\end{itemize}
	\end{itemize}

	\begin{itemize}
	\item 学习使用 Tetgen 库生成四面体剖分
\end{itemize}
	
	\subsection{实验内容}
	
	\begin{itemize}
	\item 	完成类 Simulation 中 Init 和 SimulateOnce 接口函数
\end{itemize}

	\begin{itemize}
	\item  在UEngine中添加功能,主要有
	\begin{itemize}[label=$\circ$, itemjoin=\hspace{0.5em}]
		\item 欧拉隐式方法的实现
		\item 加速模拟方法的实现
		\item  相关参数的调整交互
	\end{itemize}
\end{itemize}
	
	\section{算法设计}
	
	\subsection{背景}
	一个弹簧质点系统就是由节点及节点之间的边所构成的图(Graph),也就是网格。网格图的每个顶点看为一个质点,每条边看为一根弹簧。我们希望由前$n$帧的信息求得第$n+1$帧信息,设位移$x$,速度$v$,时间步长$h$。由牛顿定律可得:
	$$\boldsymbol x_{n+1}=\boldsymbol x_n+ h \boldsymbol v_{n+1},\boldsymbol v_{n+1}=\boldsymbol v_n+h\boldsymbol M^{-1}(\boldsymbol f_{int}(t_{n+1})+\boldsymbol f_{ext})  $$
		若记 $$ \boldsymbol y =\boldsymbol x_n + h\boldsymbol v_n + h^2\boldsymbol M^{-1}\boldsymbol f_{ext}, $$
		 则原问题转化为求解关于$\boldsymbol x$的方程$(*)$: $$ \boldsymbol g(\boldsymbol x) = \boldsymbol M(\boldsymbol x-\boldsymbol y) -h^2\boldsymbol f_{int}(\boldsymbol x) = 0, $$
		 
		
	以下分别用两种方法求解此方程。
	
	\subsection{欧拉隐式方法}

   欧拉隐式方法利用牛顿法直接求解该方程,主要迭代步骤为: $$ \boldsymbol x^{(k+1)}=\boldsymbol x^{(k)}-(\nabla \boldsymbol g(\boldsymbol x^{(k)}))^{-1}\boldsymbol g(\boldsymbol x^{(k)}). $$
   迭代的初值选为$ \boldsymbol{x}^{(0)}=y$,迭代得到位移$x$后更新速度$v_{n+1}=(x_{n+1}-x_{n})/h$。
   
   在实际的实现过程中,设置两次迭代误差小于取定的误差界时停止或迭代指定步数停止。
   
   求解时,将上式变形得到: $$ (\nabla g(x^{(k)}))(x^{(k)}-x^{(j+1)}) = g(x^{(k)}) $$
   
   对左边求导时,涉及到弹力的求导
                    $$ \nabla g(x^{(k)}) = M-h^2\nabla f_{int}(x^{(k)}) $$
   
  对于单个弹簧(端点为$\boldsymbol x_1$,$\boldsymbol x_2$),劲度系数为$k$,原长为$l$,有: $$ \boldsymbol x_1\text{所受弹力:} \boldsymbol f_1(\boldsymbol x_1,\boldsymbol x_2)=k(||\boldsymbol x_1-\boldsymbol x_2||-l)\frac{\boldsymbol x_2-\boldsymbol x_1}{||\boldsymbol x_1-\boldsymbol x_2||} $$ 
   $$ \boldsymbol x_2\text{所受弹力:} \boldsymbol f_2(\boldsymbol x_1,\boldsymbol x_2)=-\boldsymbol f_1(\boldsymbol x_1,\boldsymbol x_2), $$ 对 $$ \boldsymbol h(\boldsymbol x)=k(||\boldsymbol x||-l)\frac{-\boldsymbol x}{||\boldsymbol x||}, $$ 求导有 $$ \frac{ d \boldsymbol h}{d \boldsymbol x} = k(\frac{l}{||\boldsymbol x||}-1)\boldsymbol I-kl||\boldsymbol x||^{-3}\boldsymbol x \boldsymbol x^T. $$ 代入弹力公式得: $$ \frac{\partial \boldsymbol f_1}{\partial \boldsymbol x_1} =\frac{\partial \boldsymbol h(\boldsymbol x_1-\boldsymbol x_2)}{\partial \boldsymbol x_1}=k(\frac{l}{||\boldsymbol r||}-1)\boldsymbol I-kl||\boldsymbol r||^{-3}\boldsymbol r \boldsymbol r^T,\text{其中}\boldsymbol r=\boldsymbol x_1-\boldsymbol x_2, \boldsymbol I\text{为3$\times$3单位阵}\ $$
   
   $$ \frac{\partial \boldsymbol f_1}{\partial \boldsymbol x_2}=-\frac{\partial \boldsymbol f_1}{\partial \boldsymbol x_1}, \frac{\partial \boldsymbol f_2}{\partial \boldsymbol x_1}=-\frac{\partial \boldsymbol f_1}{\partial \boldsymbol x_1}, \frac{\partial \boldsymbol f_2}{\partial \boldsymbol x_2}=\frac{\partial \boldsymbol f_1}{\partial \boldsymbol x_1}, $$ 对所有弹簧求导并组装即可求得力的导数(组装为稀疏矩阵,矩阵为对称阵)。
	
	
	
	\subsection{加速方法}
	
	加速方法将求解方程的问题转化为了优化问题。在上述欧拉方法中,对于内力(为保守力)有: $$ \boldsymbol f_{int}(x)=-\nabla E(\boldsymbol x) $$ 
	故对方程$(*)$的求解可以转为为一个最小化问题: $$ \boldsymbol x_{n+1}=\min\limits_{x}\frac{1}{2}(\boldsymbol x-\boldsymbol y)^T\boldsymbol M(\boldsymbol x-\boldsymbol y)+h^2E(\boldsymbol x) $$
	
	引入辅助变量 $$ \boldsymbol U= { \boldsymbol d=(\boldsymbol d_1,\boldsymbol d_2,...,\boldsymbol d_s),\boldsymbol d_s\in R^3,||\boldsymbol d_i||=l_i } \text{$l_i$为第i个弹簧原长,弹簧数量为s}, $$
	
	则原问题转化为上次作业的一个Local/Global问题。
	
	$$ \boldsymbol x_{n+1}=\min\limits_{x,\boldsymbol d\in\boldsymbol U}\frac{1}{2}\boldsymbol x^T(\boldsymbol M+h^2\boldsymbol L)\boldsymbol x-h^2\boldsymbol x^T\boldsymbol J \boldsymbol d-\boldsymbol x^T \boldsymbol M \boldsymbol y $$ 
	
	其中L和J为:	
	\begin{figure}[H]
		\begin{center}
			
			\includegraphics[width=9cm,height=5cm]{L and J}
			
		 \label{jiemian.label}
		\end{center}
	\end{figure}
	
	对$\boldsymbol x $,$\boldsymbol d$ 迭代优化求得该优化问题的解: $$ \boldsymbol x \text{优化:求解方程}(\boldsymbol M+h^2\boldsymbol L)\boldsymbol x=h^2\boldsymbol J \boldsymbol d+ \boldsymbol M \boldsymbol y\text{(这里可以预分解矩阵),}\ $$ $$ \boldsymbol d \text{优化:}\boldsymbol d_i=l_i\frac{\boldsymbol p_{i_1}-\boldsymbol p_{i_2}}{||\boldsymbol p_{i_1}-\boldsymbol p_{i_2}||}\text{这里$l_i$为第i个弹簧原长,$\boldsymbol p_{i_1}$,$\boldsymbol p_{i_2}$为其两端点,} $$
	
	重复迭代过程直到收敛。
	
    \subsection{边界条件和约束}
    
	通常模拟过程中物体会有各种约束或额外条件,例如物体被固定了几个点,对某些点施加外力(如重力、浮力、风力等)。
	\subsubsection{外力条件}
	\begin{itemize}
		\item 	物体受到的外力可以直接加在模拟的外力项中,其导数为 0
	\end{itemize}
	\begin{itemize}
	\item 	对于重力,可以将其加在外力中,另一方面,重力为保守力,也可以将重力势能加在能量项中与弹性势能进行合并
   \end{itemize}
	\subsubsection{位移约束}
	这里主要考虑固定部分质点的情形,有两种方法处理:
		\begin{itemize}
		\item 	第一种方法是在每一帧中求出该点的内力,再施加与该内力大小相同,方向相反的外力,但与上一种情形不同的是,若该内力对位移导数不为 0,则该外力对位移导数也不为 0,需要将其导数考虑进去;
	\end{itemize}
	\begin{itemize}
	\item 	第二种方法为仅考虑真正的自由坐标,降低问题的维数,具体如下:
	
	将所有n个质点的坐标列为列向量$x\in R^{3n}$,将所有m个自由质点坐标(无约束坐标)列为列向量$x_f \in R^{3m}$,则两者关系为:$$ \boldsymbol x_f=\boldsymbol K\boldsymbol x,\ \boldsymbol x=\boldsymbol K^T\boldsymbol x_f+\boldsymbol b, $$ 其中$K \in R^{3m\times 3n}$ 
	为单位阵删去约束坐标序号对应行所得的稀疏矩阵,$b$ 为与约束位移有关的向量,计算为$b=x-K^{T}Kx$
	, 若约束为固定质点则$b$ 
	为常量。
	
	更新欧拉隐式方法中求解方程为: $$ \boldsymbol g_1(\boldsymbol x_f) = K(\boldsymbol M(\boldsymbol x-\boldsymbol y) -h^2\boldsymbol f_{int}(\boldsymbol x)) = 0,\ \nabla_{x_f} \boldsymbol g_1(\boldsymbol x_f) = K\nabla_{x} \boldsymbol g(\boldsymbol x)K^T,\ $$ 加速方法中优化问题中 
	$x$迭代步骤转化为求解关于$x_f$
	的方程: $$ K(\boldsymbol M+h^2\boldsymbol L)K^T\boldsymbol x_f=K(h^2\boldsymbol J \boldsymbol d+ \boldsymbol M \boldsymbol y-(\boldsymbol M+h^2\boldsymbol L)\boldsymbol b) $$
\end{itemize}


	
	
	\section{结果展示}
	
	请见演示视频
	
	\section{总结与讨论}
	
	本次实验挺有意思的,第一次接触动画,感觉很好玩。最后调试弹簧时感觉固定边的交互做的不是很好,点了好几次都没有反应。所以最后演示视频的效果不太好,但时间关系,来不及优化了。
	
	只能看出来随着弹性系数增大,弹簧的弯曲程度是在慢慢减少的这个趋势。
	
	
\end{document}